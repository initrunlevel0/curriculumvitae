\documentclass[11pt, a4paper, sans]{moderncv}
\moderncvstyle{casual}
\moderncvcolor{blue}
\usepackage[scale=0.75]{geometry}

\firstname{Putu Wiramaswara}
\familyname{Widya}

\title{Curriculum Vitae}
\address{Jalan Cempaka No. 8 Desa Banjarangkan}{Kabupaten Klungkung, Bali, 80752 Indonesia}

\mobile{+6285 956 05 6619}
\email{putu@wirama.web.id}

\homepage{wirama.web.id}

\photo[70pt][0.4pt]{photo.jpg}


\begin{document}
\makecvtitle

\section{Education}
    \cventry{2011-2015} {Sarjana Komputer / Bachelor of Computing} {Institut Teknologi Sepuluh Nopember}{Surabaya}{\textit{GPA -- 3.87}}{Lulus dengan Pujian / Cumlaude}
    \cventry{2008-2011} {Senior High School} {SMA Negeri 1}{Semarapura, Klungkung}{}{}
    \cventry{2005-2008} {Junior High School} {SMP Negeri 1}{Semarapura, Klungkung}{}{}
    \cventry{1999-2005} {Elementary School} {SD Negeri 1}{Banjarangkan, Klungkung}{}{}

\section{Undergraduate Thesis}
    \cvitem{Title}{\emph{Design and Implementation of Platform as a Service (PaaS) to Serve a Multi-Tenancy Application Platform based on Cloud Computing Principle}}
    \cvitem{Supervisor I}{Royyana M. Ijtihadie, S.Kom, M.Kom, Ph.D}
    \cvitem{Supervisor II}{Baskoro Adi Pratomo, S.Kom, M.Kom}
    \cvitem{Description}{This undergraduate thesis design and implement a web-hosting like cloud PaaS platform which support multi-platform web-based application hosting with a multi-tenancy and multi-nodes capability.}

\section{Language Skills}
    \cvitem{Balinese}{\textbf{Native} ability for Low and Mid-to-High register}
    \cvitem{Indonesian}{\textbf{Native} ability in spoken and written form}
    \cvitem{Javanese}{\textbf{Basic} Working ability for lower register}
    \cvitem{English}{\textbf{Intermediate} Working ability for Writing, Reading, Listening and Reading (IELTS: 6.5)}

\section{Computer Skills}
    \cvitem{Basic}{Mobile Apps Programming, C++, Machine Learning, Natural Language Processing, Software Testing}
    \cvitem{Intermediate}{Agile Software Project Management, UX Design, Software Architecture Design, Python, \LaTeX, .NET Framework, Java Framework, Network Design and Management, Penetration Testing, Cisco iOS, Kali Linux}
    \cvitem{Advanced}{HTML, CSS, Javascript, Node.js, Angular.js, Linux OS (GUI and Terminal), OpenOffice.org/LibreOffice}


\section{Experience}
\cventry{2009-2010}{Writing of BlankOn Linux User Guide}{BlankOn Linux}{Indonesia}{}{Wrote a book to guide BlankOn Linux users\newline{}\newline{} Skills Developed:
\begin{itemize}
    \item Indonesian Standard Spelling System
    \item Indonesianized Computing Term
\end{itemize}}

\cventry{2008-2010}{ICTKLUNGKUNG.NET team}{Dinas Pendidikan, Pemuda dan Olahraga}{Klungkung, Bali}{}{One member of the team which its responsibility was to spread the usage of Information and Communication Technology (ICT) within the education system in Klungkung Regency. The team built a radio-based WiFi network across Klungkung and Nusa Penida islands, spread the usage of Open Source Software (mainly Ubuntu Linux and OpenOffice.org) and gave many seminars and tutorials toward teachers and students\newline{}\newline{}
Skills developed:
\begin{itemize}
    \item Advanced Ubuntu and Debian GNU/Linux
    \item Advanced OpenOffice.org/LibreOffice
    \item Elementary Network and Server Configuration with Debian GNU/Linux and Mikrotik
    \item Teaching and public speaking skills
\end{itemize}}


\cventry{2009-2012}{Balinese and Javanese Unicode Font and Input Method Project}{BlankOn Linux}{Indonesia}{}{Developed two fonts for two different Indonesian traditional scripts (which are descendant of Indic Abugida) using newly drafted Unicode standard (5.0-5.2) with smartfont capability using OpenType and SIL Graphite to accommodate complex font rendering mechanism. Also developed each corresponding input method to map a regular QWERTY keyboard into its Balinese/Javanese representation\newline{}\newline{}
Skills Developed:
\begin{itemize}
    \item Familiarity with Unicode and OpenType
    \item SIL GDL Programming
    \item Input method programming in GNOME
\end{itemize}
This development had been presented in many occasional scientific research competition and had been achieved some medals :
\begin{itemize}
    \item Ranking \#1 in Scientific Research Competition (LKIR, Lomba Karya Tulis Ilmiah) at provincial level in Bali, 2009.
    \item Won a Gold Medal in Applied Science category of Student Scientific Research Olympiad (OPSI, Olimpiade Penelitian Sains Indonesia), 2010.
    \item Awardee of Student Creativity Program (Program Kreativitas Mahasiswa, PKM), 2012 batch.
\end{itemize}} 

\cventry{2012-2014}{Member of Student Union}{Himpunan Mahasiswa Teknik Computer (HMTC) ITS}{Surabaya}{}{Active member of student union which its body is responsible for internal and external student affair between/to Informatics Engineering Department of ITS. For the first year (2012), had been a member of Research and Technology Department which provide and organizes a scientific writing seminar towards Student Creativity Program (mainly) and any others writing occasion.\newline{}\newline{}
Skills Developed:
\begin{itemize}
    \item Basic Event Organizing
    \item Team Working
\end{itemize}} 

\cventry{2013-now}{Some Random Project}{Dwipasawitra}{Surabaya}{}{A small group consisting of myself (as Project Manager), Ivan (as Mathematician and Analys), Ali and Haidar (as programmer) who had been conducted series of software projects (paid project and student task project) using newly-learned and experimented language platform. We are notorious for choosing anti-mainstream method to develop a software which mainly web-based. The name of \textbf{Dwipasawitra} is came from port-manteu \emph{Dwipa} (Sanskrit for Island) and \emph{Sawitra} (High Balinese for "Friend"). \newline{}\newline{}
Skills Developed:
\begin{itemize}
    \item Python and Node.js programming
    \item Server and Client based Javascript
    \item Applied Software Management
    \item Agile, Extreme Programming and Scrum
    \item Requirement Analysis
    \item Rapid Prototyping
    \item Software Documentation Management
\end{itemize}
Some of our portfolio can be seen on \url{http://wirama.web.id/my-portofolio/}
} 

\section{Interests}
    \cvlistdoubleitem{Calisthenics}{Geography}
    \cvlistdoubleitem{Electronic Music}{Vegetarian Cooking}
    \cvlistdoubleitem{Reading e-book}{Gadget Hacking}
    \cvlistdoubleitem{Politics}{Psychology}


\end{document}


